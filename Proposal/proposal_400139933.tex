\documentclass[a4paper]{article}
\usepackage[a4paper,top=2cm,bottom=2.5cm,left=1.5cm,right=1.5cm,marginparwidth=1.75cm]{geometry}
%% Language and font encodings
\usepackage[english]{babel}
\usepackage[utf8x]{inputenc}
\usepackage{listings}

%% Sets page size and margins

\usepackage{float}
%% Useful packages
\usepackage{amsmath}
\usepackage[colorinlistoftodos]{todonotes}
\usepackage[colorlinks=true, allcolors=blue]{hyperref}
\usepackage{listings}
\usepackage{url}
\usepackage{graphicx}
\usepackage[strings]{underscore}
\graphicspath{ {./images/} }
% \DeclareGraphicsExtensions{.pdf,.jpg,.png}

%% defined colors
\definecolor{Blue}{rgb}{0,0,0.5}
\definecolor{Green}{rgb}{0,0.75,0.0}
\definecolor{LightGray}{rgb}{0.6,0.6,0.6}
\definecolor{DarkGray}{rgb}{0.3,0.3,0.3}

\title{Deep Learning to Assist Spacecraft Attitude Determination}
\author{
Stephen Scott (400139933)}
\date{\today}

\begin{document}
\maketitle

\section{Overview and Literature Search}

For this project I would like to review star-tracking algorithms used for spacecraft attitude determination. Spacecraft attitude determination is the process by which a satellite uses sensors to estimate its orientation and is important to the satellite control problem because it provides a means for the satellite to adjust its trajectory with respect to a reference\cite{adcs}. The popularity of nanosatellites like CubeSats is increasing, and typically these small platforms are much more limited both with respect to costs and available hardware when compared to conventional satellites. Attitude determination algorithms often use a fusion of data from multiple sensors to determine the best estimate of the satellite's orientation. The star tracker is a candidate for these applications; however, the star tracker can be a costly component to add to a low-budget spacecraft like a CubeSat\cite{starTrackAttDet}. Smith suggests that a star tracker can be made from budgetary hardware; however, the software to process the star tracker data in a meaningful manner is a non-trivial task and can propose a barrier to university teams developing nanosatellites. 

In the past, star trackers relied on lookup tables to identify stars; however, in recent years neural networks are being incorporated into the star tracker software stack \cite{lowCost}. The star tracker system involves an optic system used to capture light from stars. An image is developed from the optics and is typically fed through a series of algorithms including star detection and centroiding, star identification, and finally attitude estimation \cite{survey}. A crucial part of the overall algorithm is feature extraction from the star detection and centroiding. Another important aspect of the system is the onboard database which stores star features for the identification process \cite{survey}. A magnitude threshold is used to populate the database since the photodetectors can only identify stars of a certain magnitude. Also, stars that have centroids too close to one another are ignored.

One of the recent developments in star identification is a pattern based feature extraction called the correlation algorithm which compares the camera images to images in the database and maximizes a cost function. Star locations are represented as a Gaussian distribution and the images are multiplied with the database image to attain a heat map \cite{survey}.

Input images are often populated with noise so feature extraction can be difficult. In recent years, there have been developments towards using deep learning for de-noising \cite{denoise}. I would like to do some further research on this to determine if the star tracker software stack could benefit from a deep learning-based de-noising pre-processing step. These deep learning approaches to de-noising are also useful for ground-based star imaging like astrophotography.



\section{Proposal Details}

I would like to investigate the image processing techniques and algorithms used in satellite on-board computers. Specifically, I would like to pre-process the star tracker images to reduce noise and isolate star candidates. Then, I will attempt to extract some information from the pre-processed information like distance between star candidates. Finally, I would like to implement a deep learning model that can classify stars based on open-source databases. Rather than collecting the data from a star tracker directly, I will find an open source dataset of images of the night sky.

\bibliographystyle{plain}
\bibliography{bibfile}

\end{document}